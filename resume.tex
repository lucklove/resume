% !TEX TS-program = xelatex
% !TEX encoding = UTF-8 Unicode
% !Mode:: "TeX:UTF-8"

\documentclass{resume}
\usepackage{zh_CN-Adobefonts_external} % Simplified Chinese Support using external fonts (./fonts/zh_CN-Adobe/)
%\usepackage{zh_CN-Adobefonts_internal} % Simplified Chinese Support using system fonts
\usepackage{linespacing_fix} % disable extra space before next section
\usepackage{cite}
\usepackage{ulem}

\setlength{\columnsep}{25pt} %设置左右栏间距

\begin{document}
\pagenumbering{gobble} % suppress displaying page number

\name{周帅}

% {E-mail}{mobilephone}{homepage}
% be careful of _ in emaill address
\contactInfo{gnu.crazier@gmail.com}{(+86) 18202415073}{1041324091}
% {E-mail}{mobilephone}
% keep the last empty braces!
%\contactInfo{xxx@yuanbin.me}{(+86) 131-221-87xxx}{}

\section{\faUser 个人介绍}
习性宅而不萌.
\newline
喜欢底层架构, 热爱高性能计算, 乐于压榨计算机, 知识储备偏向于后端与基础架构.
\newline
除计算机技术及相关外还喜涉猎社会百科.

\section{\faGraduationCap 教育背景}
\datedsubsection{\textbf{东北大学}, 沈阳}{2013 -- 至今}
\textit{本科在读} 软件工程

\section{\faStar 创业/项目经历}
\datedsubsection{\textbf{智从科技有限公司} 沈阳}{2014年5月 -- 至今}
\role{创业}{创始人: 王艺博}
\begin{onehalfspacing}
轻反馈(iFeedback)
\newline
http://www.qingfankui.com/
\begin{itemize}
  \item 1.x: 轻反馈PC端安装包制作(安装, 升级, 卸载)
  \item 2.x: 学生偏科度分析算法实现
  \item 3.x:
  \begin{itemize} 
    \item 题目的模糊搜索算法设计与实现 
    \item 基于Angular.js实现部分页面逻辑
  \end{itemize}
\end{itemize}
\end{onehalfspacing}

\datedsubsection{\textsc{Cinatra}}{时间待确定}
\role{C++, header only}{社区开源项目, 和江南还有往事如风等合作开发}
\begin{onehalfspacing}
一套header only的web框架 
\newline
https://github.com/topcpporg/cinatra
\begin{itemize}
  \item 完成了mini单元测试框架的开发, 以之替代boost.test
  \begin{itemize}
    \item 避免boost.test臃肿的问题: 只有一个头文件, 不足200行
    \item 避免boost.test在BOOST\_REQUIRE中使用异常终止case导致的不良问题: \newline 若BOOST\_REQUIRE抛出的异常被用户代码捕获, 则case不会终止而会继续执行, 这在相当大程度上会导致整个测试被系统终止(比如访问无效内存).
    \item 满足"好莱坞原则": 每个测试用例不用关心自己会在哪里被调用, 添加测试用例也无需改动任何已有代码
  \end{itemize}
  \item 在开发过程中通过测试工具对框架性能进行评估与反馈
\end{itemize}
\end{onehalfspacing}

\datedsubsection{\textsc{ahttpd}}{时间待确定}
\role{C++, asio}{个人开源项目}
\begin{onehalfspacing}
基于boost.asio实现的一套异步的网络框架
\newline
https://github.com/lucklove/ahttpd
\begin{itemize}
  \item 实现了http server, http client, smtp client, fastcgi client
  \item 允许http server暴露io\_service给http client, smtp client或fastcgi client使用, 这样可以做到所有的子模块使用同一个io\_service从而尽量避免代码中出现同步的代码
  \item 实现了一个线程池, 防止在必须存在同步代码时占用io\_service的线程从而导致服务器卡死, 以尽可能达到高效
\end{itemize}
\end{onehalfspacing}

\datedsubsection{\textsc{PM-Game-Server}}{时间待确定}
\role{C++, lua}{商业开源项目, 和Eric合作开发}
\begin{onehalfspacing}
微信游戏"口袋妖怪"服务端
\newline
https://github.com/lucklove/PM-Game-Server
\begin{itemize}
  \item 采用了mysql作永久储存, redis作缓存, 并实现了一个lock-free的数据库连接池, 提高运行时效率.
  \item 采用lua实现游戏的逻辑, 增强扩展性.
\end{itemize}
\end{onehalfspacing}

\datedsubsection{\textsc{nua}}{时间待确定}
\role{C++, header only}{商业开源项目}
\begin{onehalfspacing}
PM-Game-Server的lua交互库, 由开源项目Selene改进而来.
\newline
https://github.com/lucklove/nua
\begin{itemize}
  \item 修复Selene采用light user data实现用户类型而导致的同时传入多个用户类型至lua无法正确识别的bug.
  \item 扩展了用户类型的接口, 允许用户自由选择传入value, reference, 还是const reference, 而不是限制用户只能使用lvalue的用户类型, 从而在增加灵活性的同时更好的支持了在lua中使用C++的多态(用户可以自主选择const和非const版本).
\end{itemize}
\end{onehalfspacing}


% Reference Test
%\datedsubsection{\textbf{Paper Title\cite{zaharia2012resilient}}}{May. 2015}
%An xxx optimized for xxx\cite{verma2015large}
%\begin{itemize}
%  \item main contribution
%\end{itemize}

\section{\faCogs\ IT 技能}
% increase linespacing [parsep=0.5ex]
\begin{itemize}[parsep=0.5ex]
  \item C++(熟练, since 2012)
  \item C(熟练, since 2011)
  \item Linux(熟练, since 2011)
  \item 异步编程, 无锁编程
\end{itemize}

\section{\faHeartO 获奖情况}
\datedline{\textit{二等奖}, 美国大学生数学建模竞赛}{2015 年2 月}
\datedline{\textit{二等奖}, 全国大学生信息安全竞赛}{2015 年8 月}
\datedline{\textit{一等奖}, 全国大学生计算机设计大赛}{2015 年8 月}

\section{\faUsers 活动经历}
\datedsubsection{\textbf{hackshanghai} top9}{2015 年11月}
hackshanghai是中国最大的大学生hackthon. 
2015年11月7日, 我和两位同伴经过24小时奋战, 
完成了一个mooc笔记插件
(https://github.com/ele828/mooc-mate),
我负责用cinatra进行服务端开发. 

\section{\faInfo 其他}
% increase linespacing [parsep=0.5ex]
\begin{itemize}[parsep=0.5ex]
  \item GitHub: https://github.com/lucklove
  \item 语言: 英语 - 熟练(CET6)
\end{itemize}

%% Reference
%\newpage
%\bibliographystyle{IEEETran}
%\bibliography{mycite}
\end{document}
