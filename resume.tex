% !TEX TS-program = xelatex
% !TEX encoding = UTF-8 Unicode
% !Mode:: "TeX:UTF-8"

\documentclass{resume}
\usepackage{zh_CN-Adobefonts_external} % Simplified Chinese Support using external fonts (./fonts/zh_CN-Adobe/)
%\usepackage{zh_CN-Adobefonts_internal} % Simplified Chinese Support using system fonts
\usepackage{linespacing_fix} % disable extra space before next section
\usepackage{cite}
\usepackage{ulem}
\usepackage{arydshln}

\newcommand{\subsectionrule}{{\vspace{-8pt}\hspace{0.5cm}\rule[1pt]{\linewidth-1cm}{0.05pt}\vspace{-8pt}}}

\begin{document}
\pagenumbering{gobble} % suppress displaying page number

\name{周帅}
\contactInfo{gnu.crazier@gmail.com}{(+86) 18202415073}{1041324091}

\section{\faUser 个人介绍}
开源软件运动的狂热支持者.
\newline
喜欢底层架构, 崇尚简洁高效, 热爱高性能计算, 乐于压榨计算机, 知识储备偏向于后端与基础架构.
\newline
除计算机技术及相关外还喜涉猎社会百科.
\newline
github小窝: https://github.com/lucklove

\section{\faGraduationCap 教育背景}
\datedsubsection{\textbf{东北大学}\ 沈阳}{2013 -- 至今}
\textit{本科在读}\ 软件工程

\section{\faStar 创业/项目经历}
\datedsubsection{\textbf{智从科技有限公司}\ 沈阳}{2014年5月 -- 至今}
\role{创业}{创始人: 王艺博}
\begin{onehalfspacing}
轻反馈(iFeedback).
\newline
轻反馈是一款学生成长分析产品, 通过轻反馈, 老师和家长可以随时沟通, 并且掌握学生成绩波动情况.
\newline
http://www.qingfankui.com/
\begin{itemize}
  \item 1.x: 轻反馈PC端安装包制作(安装, 升级, 卸载).
  \item 2.x: 学生偏科度分析算法实现.
  \item 3.x:
  \begin{itemize} 
    \item 题目的模糊搜索算法设计与实现.
    \item 基于Angular.js实现部分页面逻辑.
  \end{itemize}
\end{itemize}
\end{onehalfspacing}

\subsectionrule

\datedsubsection{\textsc{ahttpd}}{2015年3月 -- 2015年7月}
\role{C++, asio}{个人开源项目}
\begin{onehalfspacing}
基于boost.asio实现的一套异步的网络框架.
\newline
https://github.com/lucklove/ahttpd
\begin{itemize}
  \item 实现了http server, http client, smtp client, fastcgi client.
  \item 允许http server暴露io\_service给http client, smtp client或fastcgi client使用, 这样可以做到所有的子模块使用同一个io\_service从而尽量避免出现同步的代码或另开线程跑独立的io\_service.
  \item 实现了一个线程池, 可供将IO密集型操作置于其中, 以避免阻塞工作线程.
\end{itemize}
\end{onehalfspacing}

\subsectionrule

\datedsubsection{\textsc{Cinatra}}{2015年6月 -- 2015年9月}
\role{C++, header only}{社区开源项目, 和江南还有往事如风等合作开发}
\begin{onehalfspacing}
一套header only的web框架. 
\newline
https://github.com/topcpporg/cinatra
\begin{itemize}
  \item 完成了mini单元测试框架的开发, 以之替代boost.test
  \begin{itemize}
    \item 避免boost.test臃肿的问题: 只有一个头文件, 仅200行左右.
    \item 在BOOST\_REQUIRE和BOOST\_CHECK的基础上进行了扩展:
    \begin{itemize}
        \item 支持REQUIRE或CHECK失败时执行用户定义的行为.
        \item 保证REQUIRE失败时执行流不能通过后续检查点, 而BOOST\_REQUIRE抛出的异常若被用户代码捕获则无法做此保证.
    \end{itemize}
    \item 满足"好莱坞原则": 每个测试用例不用关心自己会在哪里被调用, 添加测试用例也无需改动任何已有代码.
  \end{itemize}
  \item 在开发过程中通过测试工具对框架性能进行评估与反馈.
\end{itemize}
\end{onehalfspacing}

\subsectionrule

\datedsubsection{\textsc{PM-Game-Server}}{2015年11月 -- 2016年2月}
\role{C++, lua}{商业开源项目, 和Eric合作开发}
\begin{onehalfspacing}
微信游戏"口袋妖怪"服务端.
\newline
https://github.com/lucklove/PM-Game-Server
\begin{itemize}
  \item 实现了一个lock-free stack和lock-free queue.
  \item 基于lock-free stack实现了nua::Context的无锁储存池.
  \item 基于lock-free queue实现了线程池, 将所有可能阻塞的操作(如数据库访问等IO密集型操作)置于线程池中运行, 避免工作线程被阻塞.
  \item 基于lock-free queue, lock-free stack和boost.coroutine实现了一个dispatcher, 通过dispatcher运行工作线程, 由于可能导致阻塞的操作都被置于线程池中, 工作线程只需询问操作是否完成, 若未完成就通过dispatcher将当前coroutine排队到队列末尾, 并重新选取可运行的coroutine运行, 从而保证在有任务的情况下工作线程绝不阻塞.
  \item 采用lua实现游戏的逻辑, 增强扩展性.
\end{itemize}
\end{onehalfspacing}

\subsectionrule

\datedsubsection{\textsc{nua}}{2015年12月}
\role{C++, header only}{商业开源项目}
\begin{onehalfspacing}
PM-Game-Server的lua交互库, 由开源项目Selene改进而来.
\newline
https://github.com/lucklove/nua
\begin{itemize}
  \item 修复Selene采用light user data实现用户类型而导致的同时传入多个用户类型导致lua无法正确识别的bug(https://github.com/jeremyong/Selene/issues/140).
  \item 扩展了用户类型的接口, 使之更加灵活: 用户在向lua传递用户类型时可自由选择是否使用引用以及是否使用const版本引用.
\end{itemize}
\end{onehalfspacing}


% Reference Test
%\datedsubsection{\textbf{Paper Title\cite{zaharia2012resilient}}}{May. 2015}
%An xxx optimized for xxx\cite{verma2015large}
%\begin{itemize}
%  \item main contribution
%\end{itemize}

\section{\faCogs\ IT 技能}
% increase linespacing [parsep=0.5ex]
\begin{itemize}[parsep=0.5ex]
  \item C++(熟练, since 2012)
  \item C(熟练, since 2011)
  \item Linux(熟练, since 2011)
\end{itemize}

\section{\faHeartO 获奖情况}
\datedline{\textit{二等奖}, 美国大学生数学建模竞赛}{2015 年2 月}
\datedline{\textit{二等奖}, 全国大学生信息安全竞赛}{2015 年8 月}
\datedline{\textit{一等奖}, 全国大学生计算机设计大赛}{2015 年8 月}

\section{\faUsers 活动经历}
\datedsubsection{\textbf{hackshanghai} top9}{2015 年11月}
hackshanghai是中国最大的大学生hackthon. 2015年11月7日, 我和两位同伴经过24小时奋战, 完成了一个mooc笔记插件(https://github.com/ele828/mooc-mate),我负责用cinatra进行服务端开发. 

\section{\faInfo 其他}
% increase linespacing [parsep=0.5ex]
\begin{itemize}[parsep=0.5ex]
  \item 语言: 英语 - 熟练(CET6)
\end{itemize}

%% Reference
%\newpage
%\bibliographystyle{IEEETran}
%\bibliography{mycite}
\end{document}
