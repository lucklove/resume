% !TEX TS-program = xelatex
% !TEX encoding = UTF-8 Unicode
% !Mode:: "TeX:UTF-8"

\documentclass{resume}
\usepackage{zh_CN-Adobefonts_external} % Simplified Chinese Support using external fonts (./fonts/zh_CN-Adobe/)
%\usepackage{zh_CN-Adobefonts_internal} % Simplified Chinese Support using system fonts
\usepackage{linespacing_fix} % disable extra space before next section
\usepackage{cite}
\usepackage{multicol}

\setlength{\columnsep}{25pt} %设置左右栏间距

\begin{document}
\pagenumbering{gobble} % suppress displaying page number

\name{周帅}

% {E-mail}{mobilephone}{homepage}
% be careful of _ in emaill address
\contactInfo{gnu.crazier@gmail.com}{(+86) 18202415073}{1041324091}
% {E-mail}{mobilephone}
% keep the last empty braces!
%\contactInfo{xxx@yuanbin.me}{(+86) 131-221-87xxx}{}

\begin{multicols}{2} %双栏开始

\section{\faSearch 目标职位}
\textit{C++工程师}

\section{\faGraduationCap 教育背景}
\datedsubsection{\textbf{东北大学}, 沈阳}{2013 -- 至今}
\textit{本科在读} 软件工程

\section{\faStar 创业/项目经历}
\datedsubsection{\textbf{智从科技有限公司} 沈阳}{2014年5月 -- 至今}
\role{创业}{创始人: 王艺博}
\begin{onehalfspacing}
轻反馈(iFeedback)
\newline
http://www.qingfankui.com/
\begin{itemize}
  \item 1.x: 轻反馈PC端安装包制作(安装, 升级, 卸载)
  \item 2.x: 学生偏科度分析算法实现
  \item 3.x:
  \begin{itemize} 
    \item 题目的模糊搜索算法设计与实现 
    \item 基于Angular.js实现部分页面逻辑
  \end{itemize}
\end{itemize}
\end{onehalfspacing}

\datedsubsection{\textsc{Cinatra}}{}
\role{C++, header only}{社区开源项目, 和江南还有往事如风等合作开发}
\begin{onehalfspacing}
一套header only的web框架 
\newline
https://github.com/topcpporg/cinatra
\begin{itemize}
  \item 完成了单元测试框架的开发
  \item 在开发过程中通过测试工具对框架性能进行评估与反馈
\end{itemize}
\end{onehalfspacing}

\datedsubsection{\textsc{ahttpd}}{}
\role{C++, asio}{个人开源项目}
\begin{onehalfspacing}
一套异步的网络框架
\newline
https://github.com/lucklove/ahttpd
\begin{itemize}
  \item 实现了一个侵入式的http server框架
  \item 实现了http client
  \item 实现了smtp client, 可用于发送邮件
  \item 实现了fastcgi client, 可以将请求转发给php-fpm
\end{itemize}
\end{onehalfspacing}

\datedsubsection{\textsc{PM-Game-Server}}{}
\role{C++, lua}{商业开源项目, 和Eric合作开发}
\begin{onehalfspacing}
微信游戏"口袋妖怪"服务端
\newline
https://github.com/lucklove/PM-Game-Server
\begin{itemize}
  \item 采用了mysql作永久储存, redis作缓存
  \item 游戏逻辑全部用lua实现
\end{itemize}
\end{onehalfspacing}

\datedsubsection{\textsc{nua}}{}
\role{C++, header only}{商业开源项目}
\begin{onehalfspacing}
PM-Game-Server的lua交互库
\newline
https://github.com/lucklove/nua
\begin{itemize}
  \item 采用非侵入式接口实现C++类到lua的映射
  \item 实现了lua向C++返回多值
  \item 支持lua代码中使用C++的多态
\end{itemize}
\end{onehalfspacing}


% Reference Test
%\datedsubsection{\textbf{Paper Title\cite{zaharia2012resilient}}}{May. 2015}
%An xxx optimized for xxx\cite{verma2015large}
%\begin{itemize}
%  \item main contribution
%\end{itemize}

\section{\faCogs\ IT 技能}
% increase linespacing [parsep=0.5ex]
\begin{itemize}[parsep=0.5ex]
  \item 编程语言: C++ == C > lua > Java > Python
  \item 平台: Linux
  \item 数据库: mysql, redis, memcached
  \item 虚拟化: docker, vagrant
  \item 版本控制: git, svn
  \item 编辑器: vim
  \item 编译器: gcc(g++), clang(clang++)
  \item 调试器: gdb, valgrind
\end{itemize}

\section{\faHeartO 获奖情况}
\datedline{\textit{二等奖}, 美国大学生数学建模竞赛}{2015 年2 月}
\datedline{\textit{二等奖}, 全国大学生信息安全竞赛}{2015 年8 月}
\datedline{\textit{一等奖}, 全国大学生信息安全竞赛}{2015 年8 月}

\section{\faUsers 活动经历}
\datedsubsection{\textbf{hackshanghai} top9}{2015 年11月}
hackshanghai是中国最大的大学生hackthon. 
2015年11月7日, 我和两位同伴经过24小时奋战, 
完成了一个mooc笔记插件
(https://github.com/ele828/mooc-mate),
我负责用cinatra进行服务端开发. 

\section{\faInfo 其他}
% increase linespacing [parsep=0.5ex]
\begin{itemize}[parsep=0.5ex]
  \item GitHub: https://github.com/lucklove
  \item 语言: 英语 - 熟练(CET6)
\end{itemize}

\end{multicols} %双栏结束

%% Reference
%\newpage
%\bibliographystyle{IEEETran}
%\bibliography{mycite}
\end{document}
