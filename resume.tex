% !TEX TS-program = xelatex
% !TEX encoding = UTF-8 Unicode
% !Mode:: "TeX:UTF-8"

\documentclass[UTF8]{resume}
\usepackage{zh_CN-Adobefonts_external} % Simplified Chinese Support using external fonts (./fonts/zh_CN-Adobe/)
%\usepackage{zh_CN-Adobefonts_internal} % Simplified Chinese Support using system fonts
\usepackage{linespacing_fix} % disable extra space before next section
\usepackage{cite}
\usepackage{ulem}
\usepackage{arydshln}

\newcommand{\subsectionrule}{{\vspace{-8pt}\hspace{0.5cm}\rule[1pt]{\linewidth-1cm}{0.05pt}\vspace{-8pt}}}

\begin{document}
\pagenumbering{gobble} % suppress displaying page number

\name{周帅}
\contactInfo{gnu.crazier@gmail.com}{(+86) 18202415073}{1041324091}

\section{\faUser 个人介绍}
开源软件运动的狂热支持者.
\newline
喜欢底层架构, 崇尚简洁高效, 热爱高性能计算, 乐于压榨计算机, 知识储备偏向于后端与基础架构.
\newline
除计算机技术及相关外还喜涉猎社会百科.
\newline
github小窝: https://github.com/lucklove

\section{\faGraduationCap 教育背景}
\datedsubsection{\textbf{东北大学}\ 沈阳}{2013 -- 至今}
\textit{本科在读}\ 软件工程

\section{\faStar 实习/项目经历}
\datedsubsection{\textbf{阿里巴巴}\ 杭州}{2016年6月 -- 至今}
\role{实习}{基础架构事业群-大数据\&CDN-自动化开发}
\datedsubsection{\textsc{redcoast}}{}
\role{go}{分布式日志分析平台}
\begin{itemize}
  \item 稳定性:
  \begin{itemize}
    \item 用户插件异常主动报警机制
    \item 优化内存, 解决OOM的问题, 生产环境内存占用量由37G降至3G
  \end{itemize}
  \item 易用性:
  \begin{itemize}
    \item 提供open api, 让用户可以通过AK访问红岸api
  \end{itemize}
\end{itemize}
\datedsubsection{\textsc{tesla}}{}
\role{python}{运维自动化平台}
\begin{itemize}
  \item 业务线-用户组关系重构, 由一对一改为多对多关系
  \item 配置/工作流跨业务线共享
\end{itemize}

\subsectionrule

\datedsubsection{\textsc{ahttpd}}{2015年3月 -- 2015年7月}
\role{C++, asio}{个人开源项目}
\begin{onehalfspacing}
基于boost.asio实现的一套异步的网络框架.
\newline
https://github.com/topcpporg/cinatra
\begin{itemize}
  \item 实现了http server, http client, smtp client, fastcgi client.
  \item 允许http server暴露io\_service给http client, smtp client或fastcgi client使用, 这样可以做到所有的子模块使用同一个io\_service从而尽量避免出现同步的代码或另开线程跑独立的io\_service.
  \item 实现了一个线程池, 可供将IO密集型操作置于其中, 以避免阻塞工作线程.
\end{itemize}
\end{onehalfspacing}

\subsectionrule

\datedsubsection{\textsc{Cinatra}}{2015年6月 -- 2015年9月}
\role{C++, header only}{社区开源项目, 和江南还有往事如风等合作开发}
\begin{onehalfspacing}
一套header only的web框架. 
\newline
https://github.com/topcpporg/cinatra
\begin{itemize}
  \item 完成了mini单元测试框架的开发, 以之替代boost.test
  \begin{itemize}
    \item 避免boost.test臃肿的问题: 只有一个头文件, 仅200行左右.
    \item 在BOOST\_REQUIRE和BOOST\_CHECK的基础上进行了扩展: 
    \begin{itemize}
        \item 支持REQUIRE或CHECK失败时执行用户定义的行为.
        \item 保证REQUIRE失败时执行流不能通过后续检查点, 而BOOST\_REQUIRE抛出的异常若被用户代码捕获则无法做此保证.
    \end{itemize}
    \item "好莱坞原则": 每个测试用例不用关心自己会在哪里被调用, 添加测试用例也无需改动任何已有代码.
  \end{itemize}
  \item 在开发过程中通过测试工具对框架性能进行评估与反馈.
\end{itemize}
\end{onehalfspacing}

\subsectionrule

\datedsubsection{\textsc{PM-Game-Server}}{2015年11月 -- 2016年2月}
\role{C++, lua}{商业开源项目, 和Eric合作开发}
\begin{onehalfspacing}
微信游戏"口袋妖怪"服务端.
\newline
https://github.com/lucklove/PM-Game-Server
\begin{itemize}
  \item 实现了一个lock-free stack和lock-free queue.
  \item 基于lock-free stack实现了nua::Context的无锁储存池.
  \item 基于lock-free queue实现了线程池, 将所有可能阻塞的操作(如数据库访问等IO密集型操作)置于线程池中运行, 避免工作线程被阻塞.
  \item 基于lock-free queue, lock-free stack和boost.coroutine实现了一个dispatcher, 通过dispatcher调度任务.
  \item 采用lua实现游戏的逻辑, 增强扩展性.
\end{itemize}
\end{onehalfspacing}

\subsectionrule

\datedsubsection{\textsc{nua}}{2015年12月}
\role{C++, header only}{商业开源项目}
\begin{onehalfspacing}
PM-Game-Server的lua交互库, 由开源项目Selene改进而来.
\newline
https://github.com/lucklove/nua
\begin{itemize}
  \item 修复了Selene的bug(https://github.com/jeremyong/Selene/issues/140).
  \item 扩展了用户类型的接口, 使之更加灵活: 用户在向lua传递用户类型时可自由选择是否使用引用以及是否使用const版本引用.
\end{itemize}
\end{onehalfspacing}


% Reference Test
%\datedsubsection{\textbf{Paper Title\cite{zaharia2012resilient}}}{May. 2015}
%An xxx optimized for xxx\cite{verma2015large}
%\begin{itemize}
%  \item main contribution
%\end{itemize}

\section{\faCogs\ IT 技能}
% increase linespacing [parsep=0.5ex]
\begin{itemize}[parsep=0.5ex]
  \item C++(熟练, since 2012)
  \item C(熟练, since 2011)
  \item Linux(熟练, since 2011)
\end{itemize}

\section{\faHeartO 获奖情况}
\datedline{\textit{二等奖}, 美国大学生数学建模竞赛}{2015 年2 月}
\datedline{\textit{二等奖}, 全国大学生信息安全竞赛}{2015 年8 月}
\datedline{\textit{一等奖}, 全国大学生计算机设计大赛}{2015 年8 月}

\section{\faUsers 活动经历}
\datedsubsection{\textbf{hackshanghai} top9}{2015 年11月}
hackshanghai是中国最大的大学生hackthon. 2015年11月7日, 我和两位同伴经过24小时奋战, 完成了一个mooc笔记插件(https://github.com/ele828/mooc-mate),我负责用cinatra进行服务端开发. 

\section{\faInfo 其他}
% increase linespacing [parsep=0.5ex]
\begin{itemize}[parsep=0.5ex]
  \item 语言: 英语 - 熟练(CET6)
\end{itemize}

%% Reference
%\newpage
%\bibliographystyle{IEEETran}
%\bibliography{mycite}
\end{document}
